\chapter{Considerações Finais}
    \label{chap:ConsideracoesFinais}

Ao longo da primeira etapa deste Trabalho de Conclusão de Curso (TCC), foi iniciado o estudo sobre melhorias em aplicações de Comércio Eletrônico Móvel orientando-se pela Experiência de Usuário e por critérios associados. Foram conferidos avanços no sentido de levantamentos teórico, tecnológico e metodológico. Adicionalmente, foi detalhada a proposta, apresentando uma prova de conceito, que já permite concluir que o trabalho se encontra em andamento, até mesmo no que tange à prototipação de uma típica aplicação de Comércio Eletrônico Móvel. Durante os capítulos anteriores, foram apresentadas as fundamentações teóricas, a metodologia utilizada e os primeiros resultados obtidos. Portanto, o objetivo deste último capítulo é apresentar as \nameref{ativrea} até o momento para conclusão desde trabalho. Além disso, têm-se a exposição de quais objetivos específicos foram cumpridos nesta primeira parte do estudo, em \nameref{obesp} foi alcançados nesta primeira parte do estudo. Por fim, apresentam-se os \nameref{propassos} a serem realizados na segunda parte de desenvolvimento deste estudo, e as \nameref{autora}.

\section{Atividades Realizadas}
\label{ativrea}

No escopo deste estudo, a etapa inicial foi dedicada ao desenvolvimento de atividades com o propósito principal de embasar a proposta de projeto para aprimorar a Experiência do Usuário e a Usabilidade do aplicação mais utilizada no Comércio Eletrônico Móvel no Brasil, que por razões explicadas no \hyperref[chap:Proposta]{Capítulo 5 - Proposta}, foi a Shopee. O progresso das atividades relacionadas a este estudo estão no Quadro \ref{AtivTCC1}, evidenciando a conclusão bem-sucedida de todas as tarefas planejadas. Apenas a atividade de apresentação do TCC está pendente.

\setfloatlocations{quadro}{hbtp}
\begin{quadro}
\caption{\label{AtivTCC1}Atividades Realizadas no TCC1}
\centering

\begin{tabular}{|l|l|l|}
\hline
\textbf{Atividade}               & \textbf{Status} & \textbf{Documentação} \\ \hline
Definir Tema                     & Concluído       &  Atual Estudo                     \\ \hline
Levantar do Referencial Teórico  & Concluído       &    \hyperref[chap:ReferencialTeorico]{Capítulo 2 - Referencial Teórico}                   \\ \hline
Elaborar Proposta Inicial        & Concluído       &    \hyperref[chap:Introducao]{Capítulo 1 - Introdução}                   \\ \hline
Definir Referencial Tecnológico  & Concluído       &     \hyperref[chap:ReferencialTecnologico]{Capítulo 3 - Referencial Tecnológico}                  \\ \hline
Definir Metodologia              & Concluído       &   \hyperref[chap:Metodologia]{Capítulo 4 - Metodologia}                    \\ \hline
Detalhar Proposta                & Concluído       &    \hyperref[chap:Proposta]{Capítulo 5 - Proposta}                   \\ \hline
Revisar Monografia               & Concluído       &  \hyperref[chap:ConsideracoesFinais]{Capítulo 6 - Considerações Finais}                     \\ \hline
Apresentar Trabalho à Banca      &       Pendente          &                       \\ \hline
\end{tabular}

\legend{Fonte: Autora}
\end{quadro} 

\section{Objetivos Específicos Alcançados}
\label{obesp}

Durante o decorrer da fase inicial deste estudo, objetivos específicos foram estabelecidos e descritos no \hyperref[chap:Introducao]{Capítulo 1 - Introdução}. No Quadro \ref{ObjAl} mostra o estado atual de cada um desses objetivos específicos definidos para esta pesquisa, indicando se foram "Concluídos/Alcançados". O asterisco, colocado em alguns casos ao lado do termo "Concluído", indica que esses objetivos atenderam seus status de "Concluído" de acordo com as expectativas dessa primeira etapa. Entretanto, é muito provável que ocorram evoluções para o caso da "Identificação das principais dificuldades encontradas pelos usuários ao interagirem com plataformas de comércio
eletrônico móvel" e da "Associação entre as dificuldades identificadas no objetivo anterior e as principais técnicas/práticas de usabilidade orientadas à experiência de usuário", ao longo da segunda etapa do trabalho. Nesse sentido, pode-se dizer que o cumprimento desses objetivos poderá ainda ser mais satisfatório com os resultados da segundo etapa.

\setfloatlocations{quadro}{hbtp}
\begin{quadro}
\caption{\label{ObjAl}Objetivos Específicos Alcançados no TCC1}
\centering

\begin{tabular}{|p{10cm}|p{3cm}|}
\hline
\textbf{Objetivos Específicos}                                                                   & \textbf{Status} \\ \hline
Levantamento teórico sobre o comércio eletrônico móvel no Brasil                                 & Concluído       \\ \hline
Pesquisa na literatura acadêmica referente aos conceitos de usabilidade e experiência de usuário & Concluído       \\ \hline
Identificação das principais dificuldades encontradas pelos usuários ao interagirem com plataformas de comércio eletrônico móvel &
  Concluído* \\ \hline
Associação entre as dificuldades identificadas no objetivo anterior e as principais técnicas/práticas de usabilidade orientadas à experiência de usuário &
  Concluído* \\ \hline
Aplicação de melhorias - prototipagem - em uma plataforma de comércio eletrônico móvel, com base na associação obtida com o cumprimento do objetivo específico anterior &
  Pendente \\ \hline
\end{tabular}

\legend{Fonte: Autora}
\end{quadro} 

\section{Próximos Passos}
\label{propassos}

Como próximos passos deste estudo, estão, principalmente, as melhorias que serão aplicadas no aplicativo da plataforma Shopee. Erros de usabilidade foram apontados no \hyperref[chap:Proposta]{Capítulo 5 - Proposta}, isso prejudica a experiência do usuário que poderia ser mais satisfatória. Pensando na melhoria da interface, a realização de teste de usabilidade, focado em algumas funcionalidades, está prevista, visando realmente avaliar se a solução conferida e orientada à Experiência de Usuário atenderá de forma satisfatória o público investigado.

\section{Percepções da Autora}
    \label{autora}

Até o presente momento, a autora está confiante nas contribuições do trabalho. Pode-se mencionar, nesse contexto: (i) o levantamento teórico sobre Comércio Eletrônico e Comércio Eletrônico Móvel, com maior ênfase nesse último domínio, e no âmbito nacional, ou seja, Brasil; (ii) o maior embasamento sobre os conceitos de Usabilidade e Experiência de Usuário; (iiii) o levantamento tecnológico, que tem permitido realizar as várias atividades estabalecidas para cumprimento desse trabalho; (iv) a classificação da pesquisa, e a definição do detalhamento metodológico, que tem permitido avançar em termos investigativos junto à literatura especializada, de desenvolvimento e de análise de resultados, além de conferir uma visão temporal sobre quando cada tarefa/subprocesso tem sido e será realizado; (v) a identficiação de quais plataformas de Comércio Eletrônico Móvel são mais populares no âmbito nacional, com destaque para a Shoppe; (vi) a idetntificação das principais dificuldades encontradas pelos usuários ao usarem as plataformas de Comércio Eletrônico Móvel, e (vii) a associação dessas dificuldades com as técnicas/práticas de Usabilidade orientadas à Experiência de Usuário.
\chapter[Metodologia]{Metodologia}
\label{chap:met}

Este capítulo confere os principais detalhamentos de cunho metodológicos, que guiam as várias etapas desse trabalho, com destaque para Classificação da Pesquisa, Pesquisa Bibliográfica como método investigativo; Kanban adaptado como método de desenvolvimento, e Pesquisa-ação como método de análise de resultados. Por fim, têm-se o Resumo do Capítulo.

\section{Classificação da Pesquisa}

Segundo \cite{ProjPesquisaGil}, uma pesquisa pode ser descrita como um procedimento sistemático e racional destinado a fornecer soluções de problemas. Isso é necessário quando a informação disponível é insuficiente para abordar a questão ou quando essa informação se encontra tão desorganizada que não pode ser adequadamente relacionada ao problema em questão. A condução da pesquisa requer a integração de conhecimentos existentes e a aplicação meticulosa de métodos, técnicas e outros procedimentos científicos. 

A pesquisa desenvolve-se ao longo de um processo composto por várias etapas, desde a formulação adequada do problema até a apresentação satisfatória dos resultados. Dessa forma, todo estudo científico tem início com a colocação de um problema viável, e em seguida inicia a busca por oferecer uma solução potencial, por meio de uma proposição, uma expressão verbal passível de ser comprovada como verdadeira ou falsa. Existem diversas motivações que justificam a realização de uma pesquisa, divididas geralmente em dois grandes conjuntos: motivos de ordem intelectual e motivos de ordem prática. Os primeiros derivam do desejo de adquirir conhecimento por sua própria gratificação. Os últimos têm origem no interesse de obter conhecimento para aprimorar a eficiência ou eficácia em uma determinada área \cite{ProjPesquisaGil}.

Com objetivo de desenvolver a pesquisa deste trabalho da maneira correta diante da determinado questão de problema, há a necessidade de se classificar a pesquisa no que se refere à abordagem, à natureza, aos objetivos e aos procedimentos.

A abordagem deste trabalho é definida como, predominantemente, Qualitativa. Isso ocorre, pois serão utilizadas heurísticas de usabilidade de \cite{NNGroupUI}, as quais são coletadas e analisadas de modo subjetivo, uma vez que consideram a experiência individual dos usuários em relação à interface de um sistema. Adicionalmente, ainda visando avaliar a experiência de usuário, mantém-se o viés de abordagem qualitativa, com o uso do Questionário AttrakDiff-R \cite{AttrakDiff}. Nesse caso, é verdade que se tem o uso de números concretos, na escala de 7 pontos likert, o que remete à algo de cunho quantitativo, mas no fundo, a intenção do questionário é avaliar métricas de qualidade, mantendo, predominantemente, a tendência qualitativa do estudo. Detalhes sobre essas métricas inerentes ao Questionário AttrakDiff-R, encontram-se esclarecidos no Capítulo 2 - Referencial Teórico, seção 2.3.1.

A natureza deste trabalho é classificada como Aplicada. Sendo assim, a finalidade é desenvolver conhecimentos através de aplicação prática, concentrando-se na solução de problemas específicos, e fornecendo soluções sobre o objeto em análise. Nesse sentido, o objetivo é a implantação de melhorias, orientadas à experiência de usuário, na interface de uma típica plataforma de comércio eletrônico móvel, que tenha maior circulação e popularidade entre os usuários, enquanto compram e vendem produtos e serviços. Essa implantação demandará modelagem de protótipos de alta fidelidade em ferramenta especializada, além de levantamento sobre as principais dificuldades encontras pelos usuários ao usarem a plataforma, e a associação entre essas dificuldades e as heurísticas de usabilidade centradas na experiência de usuário.

A pesquisa deste trabalho é classificada como Exploratória. De acordo com \citeonline{ProjPesquisaGil}, essas pesquisas buscam fornecer uma compreensão mais aprofundada do problema para torná-lo mais evidente ou para desenvolver hipóteses. Pode-se afirmar que o principal propósito do presente trabalho é investigar, pesquisar, visando aperfeiçoar ideias e/ou descobrir novas intuições. Isso ocorre no domínio de comércio eletrônico móvel, sendo esse acessado por diferentes tipos de dispositivos, com variadas características, procurando pesquisar se há impactos na experiência dos usuários. Em havendo, e sendo os mesmos impactos negativos, como mitigá-los usando heurísticas estabelecidas na literatura especializada.

Em termos de procedimentos, este trabalho faz uso de Pesquisa Bibliográfica e Pesquisa-ação. A Pesquisa Bibliográfica está sendo utilizada para o levantamento de referências teóricas por meio de bases de científicas. Segundo \citeonline{PesquisaAcaoTripp}, Pesquisa-ação é um processo cíclico, com o objetivo de proporcionar um melhoramento continuado, mesclando investigação e prática. Assim, ocorre planejamento, implementação, descrição e avaliação, visando melhorias contínuas. Detalhes, tanto sobre Pesquisa Bibliográfica, quanto Pesquisa-ação, são apresentados adiante, ainda nesse capítulo, seções 3.2 e 3.4.

<<links para as seções>>

\section{Pesquisa Bibliográfica}

Para elaboração deste trabalho, tem sido necessário realizar levantamentos com base em materiais da literatura especializada relacionados aos tópicos de interesse, sendo: domínio de comércio eletrônico móvel, experiência de usuário e heurísticas de usabilidade. A ideia é conferir uma base de conhecimento, que permita compreender de forma mais adequada sobre o contexto do trabalho como um todo, além de fundamentar as colocações apresentadas ao leitor dessa monografia.

Até o momento, foram utilizadas três bases cientificas para a realizar a pesquisa de materiais bibliográficos, sendo: Períodicos CAPES, IEEE e Google Acadêmico. Nessas plataformas, foram utilizadas as \textit{strings} de busca: \textit{e-commerce, m-commerce, user experience, usability} e \textit{AttrakDiff}.

% Cabe justificar o uso dos termos em inglês. 

Foram priorizados os termos em inglês, pois são conferidos muito mais referências e fontes, se comparado aos retornos conferidos com base nos termos em português. Esse quantitativo maior permite realizar uma pesquisa mais abrangente. Os próprios materiais escritos em português, comumente, referenciam os termos em inglês, sendo possível encontrar materiais escritos em português no retorno da busca.

A Tabela X resume, para cada string de busca, e combinação das mesmas, o quantitativo de materiais retornados.

Tabela X: Quantitativo de materiais retornados com as strings de busca

<<TABELA>>

Fonte: Autora

Como foram retornados vários materiais, ocorreu a necessidade de ter critérios de seleção, visando separar materiais mais aderentes ao perfil de interesse do presente trabalho. Os critérios utilizados para seleção dos materiais foram:

Os critérios utilizados para seleção dos artigos foram:

\begin{itemize}
    \item Ter sido escrito em português, inglês, alemão;
    \item O resumo ter termos chave relacionados com os tópicos de interesse desse trabalho, e
    \item Apresentar métricas relacionadas a Comércio Eletrônico (\textit{e-commerce}), Comércio Móvel (\textit{m-commerce}), Experiência de Usuário e Usabilidade.
\end{itemize}

Tendo estes critérios como base de seleção, os principais materiais que embasam o trabalho são:

Percebe-se que há materiais mais antigos, mas que são muito bem referenciados na área de Interação Humano-Computador e afins. Há ainda materiais de 2021, que conferem maior atualidade ao trabalho. De toda forma, já tem um tempo que as áreas de interesse vêm contribuindo com guias de boas práticas para se ter adequada experiência de usuário. Entretanto, de acordo com uma pesquisa prévia realizada até o momento nesse trabalho, percebe-se que as plataformas de comércio eletrônico ainda não implantam essas práticas na íntegra. Acredita-se que muito se deva à necessidade de desenvolvimento rápido dessas plataformas, para não perder mercado para os concorrentes. Afinal, desenvolver com cuidado, e de forma atenciosa, demanda tempo. Tempo, no meio comercial, á algo que custa muito caro para as empresas. 

Diante do exposto, um propósito a ser atingido nesse trabalho é justamente conferir um estudo que mostre as principais dificuldades encontradas pelos usuários, que afetam a experiência desses usuários, mesmo considerando uma plataforma muito popular e associada a um grande player do domínio de comércio eletrônico móvel. Com isso, prototipar algumas telas, conferindo um olhar da literatura, que acorda heurísticas de usabilidade, orientadas à experiência de usuário, e que mitigam tais dificuldades.

<<player em itálico>>


\section {Kanban Adaptado}



Trata-se do método utilizado por esse trabalho para condução de diferentes etapas, que vão desde a escolha do tema, às pesquisas investigativa, aplicada e avaliativa, culminando na apresentação os membros da banca.

<<falar sobre o Kanban, colocando uma ref. e mencionando o uso do Quadro Kanban, com ToDo, Doing e Done, etc...>>

Na Figura A, tem-se um exemplo do Quadro Kanban, com as colunas que orientam o fluxo de trabalho de ToDo, Doing e Done.

Figura A: Exemplo de uso do Quadro Kanban

<<Figura A>>

Fonte: Autora

Para cada etapa do trabalho, serão criados cartões específicos centrados em tarefas, em um primeiro momento, na coluna ToDo. Ao ser colocado em execução, o cartão será arrastado para a coluna Doing. Por fim, ao ser concluída a tarefa prevista no cartão, o mesmo será arrastado para a coluna Done, finalizando o processo. Isso permitirá manter o controle sobre cada etapa do trabalho.

As principais etapas do trabalho são expostas na Figura B, para o caso do TCC (primeira parte):

Definir Tema: Tarefa que foi cumprida no intuito de estabelecer um foco de estudo a ser explorado no trabalho, resultando no tema resumido no título dessa monografia. 

Definição do Referencial Teórico: Subprocesso que compreendeu: (i) levantar os principais materiais de apoio junto às bases científicas, orientando-se pelo método investigativo Pesquisa Bibliográfica, descrita anteriormente, e (ii) documentação sobre os tópicos de interesse investigados, cujos resultados encontram-se no Capítulo 2 - Referencial Teórico.

Definir Referencial Tecnológico: Tarefa que, com base nos tópicos teóricos definidos no subprocesso anterior, resultou na escrita do Capítulo 3 - Referencial Tecnológico. Têm-se, portanto, os principais suportes que apoiam a pesquisa, o desenvolvimento e outros aspectos complementares desse trabalho. Há destaque para a ferramenta Figma, a ser utilizada para prototipagem das melhorias na interface da plataforma de comércio eletrônico móvel exemplo. Além disso, cabe mencionar o uso do AttrakDiff-R, para análise dos feedbacks dos usuários, orientando-se pela perspectiva da experiência de usuário. Nesse último caso, outros detalhes serão cobertos ainda nesse Capítulo de Metodologia, na seção 3.4.

Definir Metodologia: Tarefa que compreendeu a Classificação da Pesquisa, bem como a definição dos métodos investigativo (Pesquisa Bibliográfica); de desenvolvimento (Kanban Adaptado) e de análise de resultados (Pesquisa-Ação). Como resultado dessa tarefa, tem-se o presente capítulo.

Detalhar Proposta: Tarefa visando descrever de forma mais detalhada a proposta, resultando na escrita do Capítulo 5 - Proposta.

Revisar Monografia (Primeira Parte): Tarefa que ocorreu em paralelo a todas as demais, sendo demandada sempre que uma versão mais avançada da monografia era obtida.

Apresentar Trabalho à Banca (Primeira Parte): Tarefa a ser realizada em breve, visando mostrar o trabalho aos membros da banca, recebendo e anotando seus pontos de vista.

Figura B: Etapas Primeira Parte TCC

<<Figura B>>

Fonte

Na segunda parte do TCC, Figura C, estão previstas as seguintes etapas:

Refinar Monografia: Tarefa que consistirá em refinar a monografia, conforme apontamentos dos membros da banca.

Desenvolvimento da Proposta: Subprocesso que coloca em prática as demais subtarefas inerentes aos objetivos específicos desse trabalho, com destaque para a Identificação das principais dificuldades encontradas pelos usuários ao interagirem com plataformas de comércio eletrônico móvel; e a associação entre as dificuldades identificadas e as principais técnicas/práticas de usabilidade orientadas à experiência de usuário. Aqui, será utilizado o Kanban Adaptado, já explicado anteriormente.

Análise de Resultados: Subprocesso que compreende, principalmente, a realização das etapas prevista na Pesquisa-ação, visando a aplicação de melhorias - prototipagem - em uma plataforma de comércio eletrônico móvel, com base na associação obtida com o cumprimento do subprocesso anterior "Desenvolvimento da Proposta".  Outros detalhes desse método de análise de resultados podem ser consultados em Pesquisa-ação, mais adiante nesse capítulo.

Revisar Monografia (Segunda Parte): Tarefa que ocorrerá em paralelo a todas as demais, sendo demandada sempre que uma versão mais avançada da monografia for obtida.

Apresentar Trabalho à Banca (Segunda Parte): Tarefa a ser realizada, visando mostrar o trabalho final aos membros da banca, recebendo e anotando seus pontos de vista.

Aplicar Apontamentos Finais: Tarefa que compreende a realização de refinamentos, orientando-se pelos apontamentos da banca, e visando obter a versão final da monografia.

Figura C: Etapas Segunda Parte TCC

<<Figura C>>

Fonte

\section{Pesquisa-ação}

<voltar texto que vc escreveu>>

<<termo estrangeiro - itálico>>
<<seções e caps com links para o leitor navegar>>
<<elaborar as Figuras na notação BPMN, além de print de tela do Trello, por exemplo, para ilustrar o uso do Kanban>>

A pesquisa-ação destaca-se em relação a outros métodos de pesquisa. Essa diferença não é apenas pela flexibilidade, mas também, e principalmente, devido à integração da ação dos pesquisadores e dos grupos envolvidos. Esta integração acontece em várias fases ao longo do processo de pesquisa, indo além dos elementos tradicionais de uma pesquisa convencional \cite{ProjPesquisaGil}. As etapas da pesquisa-ação são: fase exploratória, formulação do problema, construção de hipóteses, realização do seminário, seleção da amostra, coleta de dados, análise e interpretação dos dados, elaboração do plano de ação e divulgação dos resultados. Tendo em vista a condução da pesquisa deste trabalho, as etapas previstas são:

\begin{itemize}
    \item Coleta de Dados: Esta etapa consiste no levantamento de informações através de questionários. Em um primeiro momento, visando identificar quais são as plataformas mais conhecidas no domínio de comércio eletrônico móvel, elegendo uma como exemplo. Em um segundo momento, com o uso do questionário AttrakDiff-R, com o objetivo de identificar as principais dificuldades encontradas pelos usuários ao usar essa plataforma exemplo, com foco na experiência do usuário.
    \item Análise e Interpretação dos Dados: Com o levantamento de dados realizado, será possível analisar os resultados obtidos, a fim de entender quais são as dificuldades dos usuários em uma típica plataforma de comércio eletrônico móvel, de acordo com as métricas estabelecidas pelo questionário AttrakDiff-R. Com base na literatura e considerando as dificuldades identificadas, pretende-se associar as boas praticas de usabilidade, listadas pelas Heurísticas de Nielsen, e essas dificuldades.
    \item Elaboração do Plano de Ação: Com a análise dos dados concluída, inicia-se o plano de ação. Nesse caso, tem-se a elaboração de protótipos de alta fidelidade utilizando a plataforma Figma, e visando mostrar de forma clara como é possível mitigar os erros encontrados aplicando as boas práticas de usabilidade. Espera-se, com isso, que o nível de satisfação em termos de experiência de usuário seja maior.
    \item Divulgação dos Resultados: Com a finalização de cada uma das etapas da pesquisa-ação, todo o processo será documentado como parte dos resultados deste Trabalho de Conclusão de Curso como parte da conclusão da segunda parte.
\end{itemize}

\section{Resumo do Capítulo}


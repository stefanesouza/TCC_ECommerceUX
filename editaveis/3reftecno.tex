\chapter[Referencial Tecnológico]{Referencial Tecnológico} 
\label{chap:reftec}

Nesse capítulo, serão acordados as principais tecnologias e suportes, no intuito de realizar o trabalho em termos de \nameref{pesquisa}, \nameref{desenvolvimento} e \nameref{complementar}. Por fim, tem-se o \nameref{resumo}.

\section{Apoio à Pesquisa} \label{pesquisa}

Visando realizar a pesquisa, são necessárias ferramentas que viabilizem buscas por artigos e materiais de interesse, bem como a elaboração e hospedagem de questionários. 

Para o caso das buscas por artigos e demais materiais de cunho investigativo, foram utilizadas as bases científicas ACM, IEEE, bem como o Google Acadêmico disponíveis através do Periódicos Capes. Detalhes complementares sobre essas buscas constam no Método de Pesquisa, disponível no \hyperref[chap:met]{Capítulo 4 - Metodologia}.

Para o caso da elaboração e hospedagem de questionários estão sendo utilizados \nameref{GoogleForms} e \nameref{GoogleSheets}. Tais suportes estão auxiliando nos levantamentos realizados ao longo da pesquisa, seja para identificação plataformas comumente utilizadas em comércio eletrônico móvel, seja para fins de avaliação junto aos usuários.

\subsection{Google \textit{Forms}}
\label{GoogleForms}

Segundo \citeonline{GoogleForms}Pesquisas e formulários fáceis de criar para todos
Crie formulários personalizados para pesquisas e questionários, sem qualquer custo adicional. Reúna tudo em uma planilha e analise dados diretamente no Planilhas Google.

Analise as respostas com resumos automáticos
Veja gráficos com atualização de dados de resposta em tempo real. Ou abra os dados brutos com o Google Sheets para fazer uma análise ou automação mais profunda.

Crie e responda a pesquisas em qualquer lugar
Acesse, crie e edite formulários onde estiver, em telas grandes e pequenas. As pessoas podem responder à sua pesquisa em qualquer lugar, dispositivo móvel, tablet ou computador.

Crie formulários e analise os resultados com sua equipe
Adicione colaboradores como você faz no Google Docs, Sheets e Slides para criar perguntas em tempo real. Depois analise os resultados com sua equipe sem compartilhar várias versões do arquivo.

Trabalhe com dados de resposta precisos
Use inteligência integrada para definir regras de validação como resposta. Por exemplo, garanta que os endereços de e-mails estão formatados corretamente ou que números foram definidos para um determinado intervalo.

Compartilhe formulários por e-mail, links ou sites
É fácil compartilhar formulários com pessoas específicas ou um público mais amplo. É só integrar os formulários ao seu site ou postar os links nas mídias sociais.

Os Formulários Google podem receber um grande número de respostas?

Sim. O Formulários Google pode armazenar a mesma quantidade de dados que o Planilhas Google, que comporta dois milhões de células de dados. São muitos dados

\subsection{Google \textit{Sheets}}
\label{GoogleSheets}
<<falar aqui um pouco sobre Google Sheets>>

Ainda nesse sentido dos levantamentos, o questionário AttrakDiff-R, já contextualizado no capítulo anterior dessa monografia, será utilizado para realização de uma avaliação qualitativa sobre a experiência de usuário. Detalhes complementares sobre essas buscas constam no Método de Análise de Resultados, disponível no Capítulo 4 - Metodologia.


\section{Apoio ao Desenvolvimento} \label{desenvolvimento}

O trabalho tem um viés bem investigativo, junto às plataformas de comércio eletrônico móvel. Portanto, o uso dessas plataformas por diferentes dispositivos é algo relevante. Nesse contexto, está sendo utilizado os seguintes dispositivos:

<<* descrever um celular, e>>
<<* descrever um tablet ou notebook ou outro dispositivo móvel, mesmo que emulado.>>

Além disso, visando prototipar as telas com as melhorias identificadas, a partir dos feedbacks conferidos pelos usuários, será utilizada uma ferramentas de modelagem de protótipos de alta fidelidade, no caso, o Figma (REF3).

\subsection{Figma}
<<falar aqui um pouco sobre Figma>>

Há ainda a necessidade de realizar cada etapa do TCC, bem como suas tarefas e subprocessos de forma organizada e clara. Portanto, tem sido utilizado um Kanban Adaptado, com apoio do Trello (REFTrello). Demais detalhes sobre esse método de desenvolvimento constam no Capítulo 4 - Metodologia, seção 3.3.

\subsection{Trello}
<<falar aqui um pouco sobre Trello>>

\section{Apoio Complementar} \label{complementar}

Outros suportes utilizados compreendem:

* ferramentas de comunicação, tais como Telegram (REF4) e Slack (REF5). O Telegram para mensagens rápidas entre autora e orientadora. Já o Slack para manter os rastros dos orientações.

* ferramentas de modelagem de processos, como Bizagi (REF6), para modelagem de processos inerentes à apresentação da metodologia do trabalho. Exemplos desses modelos podem ser encontrados no Capítulo 3 - Metodologia.

<<falar aqui um pouco sobre Bizagi>>

* ferramentas de versionamento e hospedagem de insumos gerados ao longo do trabalho, como Git e GitHub (REF6).

<<falar aqui um pouco sobre Git e GitHub>>

<<não esquecer de mencionar, para cada suporte, as versões utilizadas e o mê e ano de último acesso aos links.>>

\section{Resumo do Capítulo} \label{resumo}

<<aqui, acrescentar uma seção para finalizar o cap., incluindo a apresentação de um Quadro, que resume os suportes tecnológicos utilizados e expostos no cap.>>

% \setfloatlocations{quadro}{hbtp}
\begin{quadro}
\caption{\label{quadro_modelo}Resumo das Ferramentas Utilizadas}



\legend{Fonte: Autora}
\end{quadro} 

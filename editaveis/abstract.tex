\begin{resumo}[Abstract]
    \begin{otherlanguage*}{english}

    \textbf{Title:} A Study of Improvements Guided by User Experience

    The advancement of mobile technologies, such as smartphones and tablets, has changed people's expectations regarding communication and lifestyle, making these devices essential in everyday life. This has driven the emergence of mobile electronic commerce (m-commerce), which has been growing faster compared to conventional electronic commerce (e-commerce). Mobile e-commerce offers advantages for both businesses and consumers, including lower costs and greater convenience. The COVID-19 pandemic has boosted the growth of online commerce, due to increased social isolation. However, m-commerce presents specific challenges related to usability on mobile devices, which raises concerns about the user experience on online buying and selling platforms accessed through mobile devices. Given this context, the present work aims to study mobile e-commerce in Brazil, providing a survey of the most used platforms, in addition to a greater understanding of Usability on these platforms, oriented to the User Experience. As a contribution to this work, high-fidelity prototypes will be developed seeking to improve possible difficulties faced by users when interacting with these platforms. It is expected that this work will provide inputs that highlight: the main difficulties of use in Mobile Electronic Commerce applications, with a User Experience bias, and the application of improvements to mitigate such difficulties.

   \vspace{\onelineskip}
 
   \noindent 
   \textbf{Key-words}: mobile e-commerce. user experience. usability. AttrakDiff questionnaire. applications.
 \end{otherlanguage*}
 
\end{resumo}

\begin{resumo}
    \textbf{Título:} Um Estudo de Melhorias Orientado à Experiência de Usuário

    O avanço das tecnologias móveis, como \textit{smartphones} e \textbf{tablets}, mudou as expectativas das pessoas em relação à comunicação e estilo de vida, tornando esses dispositivos essenciais no cotidiano. Isso impulsionou o surgimento do comércio eletrônico móvel (\textit{m-commerce}), que tem crescido mais rapidamente em comparação ao comércio eletrônico convencional (\textit{e-commerce}). O comércio eletrônico móvel oferece vantagens tanto para empresas quanto para consumidores, incluindo custos mais baixos e maior comodidade. A pandemia de COVID-19 impulsionou o crescimento do comércio \textit{online}, devido ao aumento do isolamento social. Porém, o \textit{m-commerce} apresenta desafios específicos relacionados à usabilidade em dispositivos móveis, o que levanta preocupações sobre a experiência do usuário em plataformas de compra e venda \textit{online} acessadas por meio de dispositivos móveis. Diante deste contexto, o presente trabalho tem como objetivo o estudo do comércio eletrônico móvel no Brasil, conferindo um levantamento das plataformas mais utilizadas, além de uma maior compreensão sobre a Usabilidade nessas plataformas, orientada à Experiência do Usuário. Ainda como contribuição desse trabalho, protótipos de alta fidelidade serão elaborados buscando melhorar possíveis dificuldades dos usuários ao interagirem com essas plataformas. Espera-se que este trabalho forneça insumos que evidenciem: as principais dificuldades de uso em aplicações de Comércio Eletrônico Móvel, com o viés da Experiência de Usuário, e a aplicação de melhorias para mitigar tais dificuldades.

    \vspace{\onelineskip}
    
    \noindent
    \textbf{Palavras-chave}: comércio eletrônico móvel. experiência de usuário. usabilidade. questionário AttrakDiff. aplicações.

\end{resumo}
